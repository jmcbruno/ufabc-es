\documentclass[12pt]{article}

\usepackage{sbc-template}
\usepackage{graphicx,url}
\usepackage[latin1,utf8]{inputenc}
\usepackage[brazil]{babel}
\usepackage{indentfirst} % Pacote para indentar o primeiro parágrafo depois de um título
%\usepackage[latin1]{inputenc} 
     
\sloppy

\title{Projeto de Engenharia de Software\\ Aplicativo para distribuição de informações}

\author{João Victor F. Consonni\inst{1}, Victor H. C. Leite\inst{1}, Matheus Ferreira\inst{1}, \\Matheus Milani\inst{1}, Artur L. Silva\inst{1}, Gustavo F. Fogolin\inst{1}, \\Bruno J. M. de Camargo\inst{1}, Danilo B. Cardoso\inst{1}, Daniel P. Cinalli\inst{1}, Lucas K. Kurokawa\inst{1}}

\address{Centro de Matemática, Cognição e Computação\\ Universidade Federal do ABC
  (UFABC)\\
  Av. dos Estados, 5001 - Bangú -- Santo André -- SP -- Brazil
  %\email{jconsonni, victor.costa, matheus.ferreira, milani.matheus, artur.lazarini, bruno.camargo, danilo.c@aluno.ufabc.edu.br}
  \email{\{jconsonni, victor.costa, matheus.ferreira, milani.matheus, }
  \email{artur.lazarini, gustavo.fogolin, bruno.camargo,}
  \email{danilo.c, dpcinalli, kenzo.kurokawa\}@aluno.ufabc.edu.br}
}

\begin{document} 

\maketitle

\begin{resumo} 
  Proposta de aplicativo mobile com o intuito de trazer informações e fiscalizar as barragens brasileiras, destinado às populações próximas. 
\end{resumo}


\section{Informações Gerais}
\subsection{Título do Projeto}

Dado que o principal objetivo do aplicativo será levar informação sobre as barragens para a população brasileira e assim contextualizar as pessoas sobre o status das barragens com procedimentos de segurança, dados de fiscalização e outras informações, "Barragem Transparente" \ leva em seu nome o objetivo do aplicativo. A ideia é que este nome sugira ao usuário que o aplicativo transpareça as barragens por ele citadas uma vez que o público alvo não tem domínio sobre elas.

\subsection{Definição do Problema e Justificativa do Projeto}
Em 2017, Minas Gerais possuía 5 barragens com risco de rompimento, mas a Mina do Feijão, situada em Brumadinho e que se rompeu dia 25/02, não estava nessa lista, pois não foi apontada como crítica pela Agência Nacional de Mineração (responsável pelas informações das barragens de minérios).

No fim de 2018, a Agência Nacional das Águas (ANA) divulgou um Relatório de Segurança de Barragens com dados referentes ao ano de 2017, onde podemos destacar alguns pontos importantes:

\begin{itemize}
    \item País possui 24.092 barragens com diferentes finalidades e apenas 3\% das barragens foram vistoriadas pelos órgãos fiscalizadores.
    
    \item Número de barragens com risco de rompimento subiu de 25 (2016) para 45 (2017).
    
    \item 14 incidentes ou acidentes envolvendo barragens foram reportados em 2017 no Brasil.
\end{itemize}

A Lei Nº 12.334 \cite{leiPNSB} estabelece a Política Nacional de Segurança de Barragens (PNSB) e cria o Sistema Nacional de Informações sobre Segurança de Barragens (SNISB). Segundo o Capítulo IV, Seção IV, Art. 15., a PNSB deve estabelecer programa de educação e de comunicação sobre segurança de barragem, com o objetivo de conscientizar a sociedade da importância da segurança de barragens, o qual contempla as seguintes medidas:

\begin{enumerate}
    \item Apoio e promoção de ações descentralizadas para conscientização e desenvolvimento de conhecimento sobre segurança de barragens;
    \item Elaboração de material didático;
    \item Manutenção de sistema de divulgação sobre a segurança das barragens sob sua jurisdição;
    \item Promoção de parcerias com instituições de ensino, pesquisa e associações técnicas relacionadas à engenharia de barragens e áreas afins;
    \item Disponibilização anual do Relatório de Segurança de Barragens.
\end{enumerate}

Além disso, a PNSB estabelece que barragens classificadas como de dano potencial associado alto podem ser obrigadas a elaborar um Plano de Ação de Emergência (PAE), que deve estabelecer as ações a serem executadas pelo empreendedor da barragem em caso de situação de emergência, bem como identificar os agentes a serem notificados dessa ocorrência, devendo contemplar, pelo menos:

\begin{enumerate}
    \item Identificação e análise das possíveis situações de emergência;
procedimentos para identificação e notificação de mau funcionamento ou de condições potenciais de ruptura da barragem;
    \item Procedimentos preventivos e corretivos a serem adotados em situações de emergência, com indicação do responsável pela ação;
    \item Estratégia e meio de divulgação e alerta para as comunidades potencialmente afetadas em situação de emergência.
\end{enumerate}

Com base nos pontos supracitados da PNSB, identificou-se a possibilidade e a potencial necessidade de se desenvolver um aplicativo que funcione como um meio divulgação sobre a segurança das barragens, além de acesso a material didático relacionado, disponibilização anual do Relatório de Segurança de Barragens e fácil acesso ao PAE das barragens sob a jurisdição da lei 12.334.

\subsection{Descrição do Público-Alvo}

Baseado nas utilidades do aplicativo, que enviará informações sobre barragens, podemos assumir que o principal público-alvo do projeto serão as pessoas que estão situadas ao redor das barragens, podendo assim receber o máximo de informações sobre o estado atual para que possam tomar ações preventivas em casos de alerta das barragens.

Além das pessoas que se situam ao redor, podemos também citar toda e qualquer pessoa que tenha interesse no estado da barragem, como pesquisadores e equipes de emergência para se manterem a postos para o caso de danos na barragem.

\subsection{Descrição dos Stakeholders (Interessados)}

As atribuições da PNSB são distribuídas entre várias entidades, como Órgãos Estaduais de Recursos Hídricos e Meio Ambiente e, em âmbito federal, entidades como a Agência Nacional de Energia Elétrica (ANEEL), que fiscaliza a segurança das barragens de usinas hidrelétricas; o Departamento Nacional de Produção Mineral (DNPM), que fiscaliza a segurança das barragens de rejeitos de mineração; o IBAMA, que fiscaliza a segurança das barragens de resíduos industriais; e a Agência Nacional de Águas (ANA), que fiscaliza a segurança das barragens de usos múltiplos \cite{SNISB}.

Qualquer uma dessas entidades poderia ser uma stakeholder em potencial, contudo, a ANA tem um papel central entre elas, promovendo a articulação entre esses órgãos e coordenando o SNISB, além de divulgar o Relatório de Segurança de Barragens anualmente. Isso faz com que a ANA seja a principal stakeholder do projeto.

O empreendedor, ou seja, o agente privado ou governamental com direito real sobre a barragem e o reservatório, é o responsável pela elaboração do PAE, o que poderia classificá-lo como um potencial stakeholder. Contudo, vale ressaltar que, pela lei 12.334 \cite{leiPNSB}, o PAE deve estar disponível no empreendimento e nas prefeituras envolvidas, bem como ser encaminhado às autoridades competentes e aos organismos de defesa civil, o que implica que os órgãos fiscalizadores o possuem e que ele é público, descartando a necessidade de envolvimento do empreendedor como stakeholder.

\subsection{Descrição do Perfil da Equipe}
%Tabela com nomes e funções

%Descrição do time
\begin{description}
\item[Matheus Ferreira (Scrum Master),] \noindent Técnico em Eletrônica pelo SENAI São Paulo. Graduando no Bacharelado em Ciência da Computação na UFABC, trabalhou durante dois anos na UFABC Jr. passando pelos cargos de Diretor Administrativo e Financeiro,. Hoje trabalha no Banco Itaú como Analista Jr. desenvolvendo aplicações em Java e JavaScript, atreladas à metodologia de processos BPM. Tem experiência com automação de processos, modelagem de negócios, análise de dados e implementação de software.
\subitem \textbf{MBTI:} ISTJ;
\subitem \textbf{Habilidades Técnicas e Ferramentas: } DevOps, BPM, Java, JavaScript.
\newline
\item[João Victor F. Consonni (Product Owner),]\noindent Formado em Engenharia Aeroespacial pela Universidade Federal do ABC, com graduação-sanduíche pela Arizona State University (ASU). Hoje cursa Ciência da Computação na UFABC. Tem experiência em pesquisa  e desenvolvimento de redes modulares de sensores de áudio, também com gerenciamentos de projetos e riscos, gestão de cadeia logística e otimização de recursos humanos. Trabalha no suporte ao desenvolvimento ABAP/SAP como parte do seu estágio no setor de TI da Pirelli Comercial. 
\subitem \textbf{MBTI:} INFP;
\subitem \textbf{Habilidades Técnicas e Ferramentas:} Excel, Matlab, SolidWorks, Fortran, C, Java.
\newline
\item[Artur Lazarini Silva,] \noindent Graduando em Ciência da Computação na UFABC e trabalha como Técnico em Eletrônica na Nice Brasil - TheNiceGroup. Tem experiência em sistemas de acesso à residências, condomínios e empresas, e também no desenvolvimento de hardware e software embarcado.
\subitem \textbf{MBTI:} ISTP;
\subitem \textbf{Habilidades Técnicas e Ferramentas:} Java, C, C\#, Design de Placas de Circuíto Impresso (PCB);
\newline
\item[Bruno J. M. de Camargo,]\noindent Técnico em Informática pela Escola Técnica Estadual de São Roque (ETECSR). Graduando nos bacharelados em Ciência da Computação e Neurociência pela UFABC. Tem experiência com Social Listening, processos de Bussiness Intelligence e Psicofísica. Atualmente faz estágio em Quality Assurance no CTIO - TIM Brasil. 
\subitem \textbf{MBTI:} INTP;
\subitem \textbf{Habilidades Técnicas e Ferramentas:} Excel, Python, Illustrator, Photoshop.
\newline
\item[Daniel Cinalli,] \noindent Cursa Ciênca da Computação e Engenharia de Instrumentação, Automação e Robótica pela UFABC. Formado em Ciência e Tecnologia pela mesma instituição. Possui perfil analítico e facilidade na resolução de problemas. Hoje desenvolve atividades de estágio na Sabesp. 
\subitem \textbf{MBTI:} ISFJ;
\subitem \textbf{Habilidades Técnicas e Ferramentas:} Excel, Java, Python.
\newline
\item[Danilo Brandão,] \noindent Técnico em Informática pela Escola Técnica Estadual "Dr. Emílio Hernandez Aguilar". Formado em Ciêcia e Tecnologia e graduando em Ciência da Computação pela UFABC. Tem experiência em Gestão Pública e hoje faz estágio na Belem Pneus Ideal de Franco da Rocha, São Paulo.  
\subitem \textbf{MBTI:} INTJ;
\subitem \textbf{Habilidades Técnicas e Ferramentas:} Java, C\#.
\newline
\item[Gustavo F. Fogolin,] \noindent Cursa Ciência da Computação na UFABC. Tem experiência com analise dados, mineração de dados e automação de processos, com enfoque na integração de VBA com Excel e Outlook, também na gestão e construção de Indicadores de Performance. Atualmente é Analista de Negócios Jr. no Banco Itaú. 
\subitem \textbf{MBTI:} INTP;
\subitem \textbf{Habilidades Técnicas e Ferramentas:} Excel, Java, C, C++, JavaScript;
\newline
\item[Lucas Kenzo Kurokawa,] \noindent Estuda Ciência da Computação na UFABC e atualmente é estagiário no Departamento de Pesquisa e Inovação - Inteligência Artificial do Banco Bradesco. Também é Técnico em Informática pela Escola Estadual Técnica "Lauro Gomes". Tem experiência na construção de provas de conceitos e rotinas de pesquisa de Inteligência Artificial.  
\subitem \textbf{MBTI:} ISFJ;
\subitem \textbf{Habilidades Técnicas e Ferramentas:} Python, Java, SQL, Administração de Redes, C, C\#, JavaScritpt;
\\
\item[Matheus Milani,] \noindent Graduando em Engenharia da Informação pela UFABC. Possui experiência em desenvolvimento WEB   com framework MVC. Atualmente é estagiário na Codus Tecnologia. 
\subitem \textbf{MBTI:} ISFJ;
\subitem \textbf{Habilidades Técnicas e Ferramentas:} Ruby on Rails, Ruby, AutoCAD, C, Java, JavaScript, Haskell, Excel, MVC;
\\ 
\item[Victor H. C. Leite,] \noindent Graduando em Engenharia da Informação pela UFABC. Possui experiência em pesquisa com enfoque no estudo de reversibilidade de processos em computação quântica, também no desenvolvimento de aplicações em C/C++ para sistemas embarcados. Atualmente faz estágio em desenvolvimento de software na Atech - Negócio em Tecnologia S/A.
\subitem \textbf{MBTI:} INTJ;
\subitem \textbf{Habilidades Técnicas e Ferramentas:} C, C++, Java, Excel, Python, Matlab.
\end{description}


\subsection{Descrição das Ferramentas a Serem Utilizadas}
%Precisa ajustar para identar a esquerda
\begin{description}
\item[Trello,]\noindenté uma ferramenta para gerenciamento de projetos que pode ser ajustada de acordo com as necessidades do usuário. Nós o utilizaremos para organizar as tarefas da Sprint, onde teremos colunas dividindo os cartões (que representam as tarefas) em “A fazer”, “Fazendo” e “Feito”, além de ser possível identificar quais pessoas da equipe estão em cada tarefa.
\\
\item[Google Drive,]\noindenté uma plataforma em nuvem que permite a criação de um repositório que será compartilhado por toda a equipe, onde serão colocados arquivos e documentos relevantes para o projeto, além disso, também fornece a ferramenta Google Docs, que será utilizadas para desenvolvimento colaborativo de diversos documentos e relatórios do projeto.
\newline
\item[Overleaf - LaTeX,]\noindent é uma ferramenta que possibilita a edição de textos usando Latex, onde toda a equipe pode editar o documento simultaneamente, similar ao Google Docs, de modo a uniformizar as formatações dos textos para toda a equipe. Será utilizado para gerar as versões finais dos relatórios e documentos do projeto.
\newline
\item[Android Studio,]\noindent é a IDE principal para desenvolvimento de aplicativos para Android, com diversos recursos para aumentar a produtividade, será utilizada para o desenvolvimento da aplicação, na linguagem Java.
\newline
\item[Lucidchart,]\noindent é uma plataforma online para criação de diagramas, onde é possível trabalhar colaborativamente. Será utilizada para produção dos diagramas UML e outros que forem necessários durante o projeto.
\newline
\item[Bizagi,]\noindent  é uma plataforma digital de negócios, onde é possível modelar processos com ferramentas totalmente baseadas em notação BPMN capazes de oferecer simplicidade e eficiência. Será utilizado para fazer diagramas e modelos de processos.
\newline
 \item[WhatsApp,] \noindent  é um aplicativo de comunicação com grande difusão no contexto dos usuários brasileiros. Possibilita a troca de informações de forma geral pela equipe, servindo como principal meio de comunicação entre os membros, devido a praticidade, confiabilidade e rapidez com que todos possam receber as informações.
\end{description}

\subsection{Entregas}
Por ser a primeiro contato entre as pessoas, e ainda que com Stories/tasks de planejamento, foi introduzido o Scrum e atentamos a nos adaptar no melhor estilo da metodologia, aproveitando sua característica “ágil”, fazendo um começo de trabalho mais leve e aberto para todos, pois este método irá nos permitir mudar o percurso do trabalho, caso haja necessidade durante o planejamento das próximas Sprints.

Para controlar o fluxo de atividades de cada membro da equipe, utilizamos como pontapé inicial o “Trello”, a fim de listar e distribuir stories entre o time. Devido a seu painel ser compartilhado entre a equipe, o Kan Ban foi e será extremamente importante nas entregas, para manter a organização nesse início de trabalho, o que será importante para implementar e enriquecer uma metodologia ágil cada vez mais forte entre nós.

Foram pesquisadas ferramentas ágeis para mensurar e identificar o andamento dos cards e encontrar possíveis “gargalos” que venham a acontecer. Para isso, por exemplo, irá ser implementado o gráfico BurnDown, para acompanhamento do trabalho, em comparação com o que foi planejado para a rodada de Sprint na cerimônia de Planning anterior.

A equipe está empenhada em realizar o projeto com o método ágil Scrum. Por se tratar de um projeto de desenvolvimento de aplicação, intimamente ligado a tecnologia da informação, sabemos que estamos suscetíveis a mudanças durante este percurso (a adequação ao mercado e aos Stakeholders) e para isso, nada melhor que poder refinar e dividir nossas entregas em pequenos períodos de tempo, a fim de gerar mais valor para o cliente e satisfação de todos que estão participando na equipe.

\bibliographystyle{sbc}
\bibliography{referencias}

\end{document}
