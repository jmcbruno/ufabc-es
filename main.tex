\documentclass[12pt]{article}

\usepackage{sbc-template}
\usepackage{graphicx,url}
\usepackage[latin1,utf8]{inputenc}
\usepackage[brazil]{babel}
\usepackage[nonewpage]{imakeidx}

\sloppy

\title{Modelagem de aplicativo para distribuição de informações}

\author{João Victor F. Consonni\inst{1}, Victor H. C. Leite\inst{1}, Matheus Ferreira\inst{1}, \\Matheus Milani\inst{1}, Artur L. Silva\inst{1}, Gustavo F. Fogolin\inst{1}, \\Bruno J. M. de Camargo\inst{1}, Danilo B. Cardoso\inst{1}, Daniel P. Cinalli\inst{1}, Lucas K. Kurokawa\inst{1}}

\address{Centro de Matemática, Cognição e Computação\\ Universidade Federal do ABC
  (UFABC)\\
  Av. dos Estados, 5001 - Bangú -- Santo André -- SP -- Brazil
  %\email{jconsonni, victor.costa, matheus.ferreira, milani.matheus, artur.lazarini, bruno.camargo, danilo.c@aluno.ufabc.edu.br}
  \email{\{jconsonni, victor.costa, matheus.ferreira, milani.matheus, }
  \email{artur.lazarini, gustavo.fogolin, bruno.camargo,}
  \email{danilo.c, dpcinalli, kenzo.kurokawa\}@aluno.ufabc.edu.br}
}

\begin{document} 

\maketitle

\begin{abstract} 
  
\end{abstract}

\begin{resumo} 
   
\end{resumo}
\section{Introdução}
O principal foco de um software baseado em sistemas estruturados é de possibilitar aos desenvolvedores organizarem e visualizarem seus softwares de maneira ampla, fazendo uma abstração das necessidades de seu cliente, permitindo uma visualização do software como um todo antes mesmo do inicio de sua implementação. Esta visão possibilita o desenvolvimento de sistemas maiores e mais complexos, diminuindo custos e aumentando potencialmente evolução do sistema.

O Unified Modeling Language (UML), ou linguagem unificada de modelagem em tradução livre, consistem de diferentes métodos de modelagem de sistemas amplamente utilizados por engenheiros de software, para organizar e estruturar os requisitos do sistema, a fim de gerar visualizações de um meta modelo do sistema funcional. Estes meta modelos por sua vez modelam as  não mitigam todos os erros inerentes ao sistema que possam ocorrer durante a fase de implementação do software, entretanto reduzem muito a quantidade e seriedade de problemas que poderiam ocorrer sem uma modelagem adequada.

O documento de requisitos, gerado a partir de interações com o cliente final e a equipe de desenvolvimento, é o passo inicial para o inicio da modelagem do software. O documento de requisitos contem todos os aspectos funcionais, não funcionais e de domínio do software. Devido a sua linguagem simples, permite que o engenheiro de software entenda e apresente ao seu cliente e desenvolvedores de software, pontos guias para o entendimento do funcionamento do sistema. A partir dessa modelagem, os envolvidos no projeto podem observar as conexões entre requisitos, permitindo ao cliente uma ideia geral e aos desenvolvedores a lógica miníma para o desenvolvimento do código.

Os engenheiros de software que desenvolvedores estes modelos necessitam periodicamente revisar e melhorar seus modelos, a fim de garantir que as necessidades do cliente estejam de acordo com o documento de requisitos, sempre focando em unificar, padronizar e reutilizar classes de elementos que possam ser evoluídos ou melhorados em outros projetos.

\section{...}%Conteudo
\section{Discussão}
\section{Conclusão}
\end{document}